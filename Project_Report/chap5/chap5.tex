\chapter{Conclusion} \label{chap5}
\thispagestyle{empty}


\vspace*{40 ex}
%============================================================


This research proposed a smart air pollution monitoring system that constantly keeps track of air quality in an area and sends data into cloud storage and also displays the air quality measured on Flutter Application in real time. The environment data which this monitoring system is sensing, contains the label of various gases such as carbon monoxide(CO), carbon dioxide($CO_{2}$), ammonium($NH_{4}$), Liquefied petroleum gas(LPG), air quality index(AQI) and smoke and other parameters like temperature, humidity and many more. The beauty of this proposed system is that we can see real time data of the location where this system is setup through all around world.

The second important feature of this monitoring system, it shows weather forecast of next 5 days from current time. The forecasting contains temperature, humidity and pressure and it does six predictions in a day(i.e after every 3 hours).

The third important feature of this monitoring system, it alarms to the people for the weather condition both depend on current real time data and also feature prediction if the label goes up to decided threshold then it alarms to it's user.

The four important feature of this monitoring system, it shows a detail analysis of data by plotting graph to each and every part of data with respect to time.

The fifth and last important feature of this monitoring system, it shows current data of weather API and next 5 days weather forecast depend on Geolocation(i.e latitude and longitude).

The system helps to create awareness of the quality of air that one breathes daily. This monitoring device can be delivered real-time measurements of air quality in national level. If air quality is less than 500 ppm then it is fresh air and if it is between 1000 ppm to 2000 ppm then it is poor air quality we should open the windows of the room and at last if it is greater than 2000 ppm then it is danger the area is very much polluted. When we start sensing air and noise pollution the area where we placed our air and sound come under the range where air quality is in between 200ppm to 750 ppm it comes under fresh air quality region.



\section{Future direction}
The project is intended victimization structured modeling and is ready to supply the
required results. It is with success enforced as a true Time system with bound
modifications. Science is discovering or making major breakthrough in varied fields, and thus technology keeps dynamic from time to time. Going more, most of the units is fictional on one in conjunction with microcontroller so creating the system compact thereby creating the present system simpler.

To make the system applicable for real time functions parts with larger vary must be
enforced. This system is any enlarged to observe the developing cities and industrial zones for pollution monitoring. To safeguard the general public health from pollution, this system provides associate economical and low price resolution for continuous observance of atmosphere.

In future we may be extended to this system include features such as:
\begin{itemize}
\item Shifting from Raspberry PI to some low cost microcontroller
\item Adding more sensors to get more detailed air quality
\item Adding new model so that it can predict more features such as windspeed, airdirection, etc.
\item Adding more features in flutter applications such as bar charts, histogram analyzes data in more detail
\end{itemize}


