\chapter{Literature Review}
\label{chap2}
%##########################################


\vspace*{45 ex}
%##########################################



\paragraph*{Outline:} This chapter presents the following:
\begin{enumerate}
\setlength{\itemsep}{-0.3em}
\item Efficient Noise and Air Pollution Monitoring System
\item IOT Based Air Pollution Monitoring System 
\item IOT Based Air and Noise Pollution Monitoring using Raspberry Pi
\item Air Pollution Monitoring System Using Mobile GPRS Sensors
\item Embedded device for measuring noise and air levels in atmosphere
\item IOT Based Air and Noise Pollution Monitoring in Urban and Rural Areas
\end{enumerate}


%+++++++++++++++++++++++++++++++++++++++++++++++++++++++++++++++++
\newpage
% \section{Introduction}
% Research in science and technology has played a vital role in improvising human life at great extent. With the development of instrumentation and computation facilities, research on frontier areas~\cite{bohmova+03} has gone manifold. The discussion on the frontier areas of research~\cite{collins03} in inter- disciplinary subject has always yielded novel ideas and collaborative research. In view of this, the First International Conference on Smart Technologies in Computer and Communication~\cite{choudhury07} (SmartTech-2017) is meticulously planned to muster innovative ideas from researchers, scientists, academicians, Industry professionals and students. The aim of the conference is to provide a common platform to share and discuss the novel ideas, technologies and research findings to promote interdisciplinary research and to ignite young brains.

% \begin{figure}[!ht]
% \centering
% \includegraphics[scale=0.9]{car2.png}
% \caption{\label{img2} Image caption}
% \end{figure}

% Description of Figure~\ref{img2}. .... 
% Research in science and technology has played a vital role in improvising human life at great extent. With the development of instrumentation and computation facilities, research on frontier areas has gone manifold. The discussion on the frontier areas of research in inter- disciplinary subject has always yielded novel ideas and collaborative research. In view of this, the First International Conference on Smart Technologies in Computer and Communication (SmartTech-2017) is meticulously planned to muster innovative ideas from researchers, scientists, academicians, Industry professionals and students. The aim of the conference is to provide a common platform to share and discuss the novel ideas, technologies and research findings to promote interdisciplinary research and to ignite young brains~\cite{bharati+09}.

% \section{SomayyaMadakam, R. Ramaswamy “Internet of Things”(2015)}
% This paper deals with that future is Internet Of Things, which will modify the real world things into intelligent practical things. Internet of Things works on the principal to unite everything in our world under same infrastructure, giving us not only control of object around us, but also keeping us informed of the condition of the object. The objective of this paper is to deliver an abstract of Internet of Things, architectures and essential technologies and their researchers, who wants to do research in the field of Internet of Things. 


\section{AnjaiahGuthi “Efficient Noise and Air Pollution Monitoring System” (2016)}
This paper deals with smart sensor network that are an emerging field of research which combines many challenges of computer science, wireless communication and electronics. In this research paper a solution for monitoring the noise and air pollution levels in industrial environment or any other area of interest using wireless embedded computing system is proposed.

The running representation of the initiated hardware is evaluated using original implementation, consisting of Arduino hardware support package. The hardware is tested for two or three parameters like noise, CO2 and radiation levels with respect to the normal behavior levels or given specification which provide a control over the pollution monitoring to make the environment smart.



\section{PalaghatYaswanthSai(2017)}
This document deals with the IOT Based Air Pollution Monitoring System in which we will gather the air value in PPM(Parts Per Million) as well as sound value in decibel over a web using internet and with the help of Wi-Fi module and will trigger an alarm when the air quality goes down further a certain level, means when there are sufficient amount of harmful gases are present in the atmosphere like CO2, smoke, alcohol, benzene and NH3. It will show the air quality in PPM on the LCD 16x2 display and as well as on web so that we can gather information very easily. In this MQ-135 gas sensor is used which is the best choice for monitoring air quality as it can detect most harmful gases and can measure their value efficiently. In this project, you can gather the level of air pollution and noise from anywhere using your computer or mobile. System can be set anywhere in the world and we can also activate some objects like for example when pollution goes beyoud some set level we can switch on the exhaust fan.

\section{Uppugunduru Anil Kumar, G Keerthi (2017) IOT Based Air and Noise Pollution Monitoring using Raspberry Pi}

A systematic environment monitoring hardware is required to monitor the conditions in case of exceeding the level of parameters(e.g., noise, CO2). When the things like environment equipped with sensor devices, micro-controller and various software applications becomes a self-protecting and self-monitoring environment. People need different types of monitoring hardware which are depends on the type of data monitored by the sensor. Event Detection based and Spatial Process Estimation are the two categories to which applications are classified. Initially the sensor devices are deployed in environment  to detect the parameters(e.g., noise, CO, and radiation levels etc.) which the data acquisition, computation and controlling action e.g., the variations in the noise and CO level with respect to the specified levels). Sensor devices are placed at different location to collect the data to predict the behaviour of a particular area of interest. The main focus of the this document is to model and implement an efficient monitoring
hardware through which the required values are measured remotely using web and the
information gathered from the sensors are stored in the cloud and to project the estimated
trend on the web browser.

\section{Ashvini S .kale in Air Pollution Monitoring System Using Mobile GPRS Sensors}
This paper contains brief introduction to vehicular pollution. The proposed system consists of a transmitter and receiver part. The transmitter part is integrated with single chip micro-controller, air pollution sensors array a General Packet Radio Service Modem (GPRS Modem) and a Global Positioning System Module (GPS Module) for transmitting the information.

\section{L.Ezhilarasi, K.Sripriya, “A SYSTEM FOR MONITORING AIR AND SOUND
POLLUTION USING ARDUINO CONTROLLER WITH IOT TECHNOLOGY”
(2017)}

The document contains embedded device for measuring noise and air levels in atmosphere and to make the environment in intelligent or interactive with things. The proposed model is flexible and distributive in nature to measure the environmental parameters. The architecture is developed for noise and air pollution monitoring Smart sensor network are the coming field of research and investigation which may lead to many challenges of computer science, wireless communication and electronics.

\section{Dr. Siva yellampalli “IOT Based Air and Noise Pollution Monitoring in Urban
and Rural Areas, Important Zones like Schools and Hospitals in Real Time” (2017)}

Today's major environment \& public issue is air pollution. According to the report of World Health Organization(WHO), air pollution is significant risk factor for multiple health conditions including skin \& eye infection, irritation of nose, throat \& eyes. It also causes serious condition like heart disease, lung cancer difficulty in breathing \& many. Parking management is also main public issue in most of metropolitan cities and that is also the reason of many problems. The main objective of project is by using various sensors, GSM/GPRS module and Cloud/server to design an efficient and remote system to monitoring the level of various pollutants causing pollution and to minimize the effect of these parameters without affecting the natural environment and provide live updates to avoid conflicts.  
