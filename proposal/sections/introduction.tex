
\chapter{Introduction}

\section{Overview}

The air pollution detector is an application of the Internet of Things (IoT) and cybersecurity. This thesis project is a combination of both hardware and software that enables the collection of information about the air pollution level. This project will be having a feature of collecting data about air pollution levels after every certain period and send on the cloud with the help of Arduino Uno Ethernet shield. And Machine Learning algorithms will analyzer those data and tell the level of air pollution.\\


\begin{center}
	\colorbox{green!40}{{\Huge \bfseries IOT + Cybersecurity + ML = Project}}
\end{center}



\section{Goals}

I’m proposing this thesis project to fulfill the following six goals.

\begin{enumerate}
	\item {To collect information about the air pollution level after every certain period.}
	\item {To whatever data has been collected, encrypt those data so that no one will be able to understand apart from the sender and receiver during communication.}
	\item {To send encrypted information about the air pollution levels on the cloud.}
	\item {To decrypt the information of air pollution level by Machine Learning(ML) algorithms and analyze the level of air pollution on the cloud.}
	\item {To send the information about air pollution level on Android smartphone whatever has been analyzed by Machine Learning(ML) algorithms whenever the client needs.}
	\item {To send encrypted information to Android clients form the cloud and Android client will decrypt the information.}
	
\end{enumerate}



\section{Specifications}

There are following specifications of this air pollution detector.

\begin{enumerate}
	\item {Real time access.}
	\item {Real time result by Machine Learning algorithms.}
	\item {Secure communication both at sender and receiver.}
	\item {User friendly and easy to access.}
	\item {Low cost.}
\end{enumerate}
